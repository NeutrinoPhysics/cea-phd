%\chapter{Modern Observational Cosmology}
%\label{chap:cosmo}
%\tableofcontents
%\vskip 1cm

\vspace*{3pc}
\begin{center}
\begin{minipage}{0.7\linewidth}
\hrule
\vspace{8pt}
{\huge\guillemotleft} ~To see a world in a grain of sand\\
And a heaven in a wild flower\\
Hold infinity in the palm of your hand\\
And eternity in an hour. {\huge\guillemotright} \\
\vspace{2pt}
\begin{flushright}
--- \textsc{William Blake}, \textit{Auguries of Innocence} (1950)
\end{flushright}
\vspace{8pt}
\hrule
\end{minipage}
\end{center}
\vspace{3pc}


%\section*{Introduction}

\begin{intro}
{\color{purple}A}s I've layed out in the previous chapter, neutrinos are very weakly interacting and quasi-massless. One could easily rule them out as irrelevant to the field of cosmology. However, in this current chapter and the next, I lay out how preponderant a role they play on the Universe's background density and on the formation of large scale structures. Synthesized in the early Universe, neutrinos were produced in large enough quantities to essentially constitute an additional radiation component, along with photons, which I detail in Sec.~\ref{sec:thermal}. Moreover, neutrinos are thus far the only fundamental particle we've detected that have all four pre-requisite characteristics of a dark matter candidate: they are stable, electrically neutral, are not sensitive to the electromagnetic fundamental interaction, and have mass. In Sec.~\ref{sec:inventory}, I recap the main components in the Universe: baryons, dark matter, neutrinos and photons; neutrinos contributing as either a matter or a radiation component at different epochs. In Sec.~\ref{sec:geometry}, I introduce the framework of standard observational cosmology, in which I derive the relevant physical laws from first principles. This chapter is heavily influenced by \textsc{Fundamentals of Cosmology}~\citep{JamesRichBook}, \textsc{The Early Universe}~\citep{KolbTurnerBook}, \textsc{Modern Cosmology}~\citep{DodelsonBook} and many of my academic notes from lectures including \textit{Barbara Ryden}'s (Ohio State U.) lectures from the 2016 cosmology summer school\footnote{\url{http://indico.ictp.it/event/7626/overview}} at the ICTP in Trieste, and \textit{\'Eric Gourgoulhon}'s (LUTH, Obs. Paris) lectures on General Relativity\footnote{\url{https://luth.obspm.fr/~luthier/gourgoulhon/en/master/index.html}} at the A{\&}A Masters program\footnote{\url{http://ufe.obspm.fr/Master/Master-2-Recherche/}}. 
\end{intro}
\vspace{1.5pc}