%\newcommand{\PhDTitleEN}{Determining the Mass of Cosmological Neutrinos using Lyman-alpha Forests} %% Title
%\newcommand{\PhDkeywordsEN}{Neutrinos, Dark Matter, Lyman-alpha, Cosmology, SDSS/BOSS} %% 3-6 Keywords
%\newcommand{\PhDsumEN}{In the work presented in this thesis, I use the power spectrum of the transmitted flux in the Lyman-alpha (Ly-$\alpha$) forest of distant quasars to constrain the mass of cosmological neutrinos in the context of four seperate projects. Neutrinos leave a signature imprint on large scale structures in the Universe through their free-streaming, which manifests as a deficit of matter density fluctuations on typical length scales that are inversely proportional to their rest mass. This typical free-streeming scale, of order a few Mpc, can be probed by Ly-$\alpha$ forests which are imprints of the neutral atomic Hydrogen density along the background quasar's line-of-sight. I use the Ly-$\alpha$ flux power spectrum from mainly two large scale structure surveys: the $13,821$ low-resolution quasar spectra from the ninth data release of SDSS (BOSS) in 12 redshift bins from $\langle z \rangle = 2.2$ to $4.4$; and the $100$ high-resolution quasar spectra from the XQ-100 survey (of the VLT's XShooter spectrograph) in 3 redshift bins, $\langle z \rangle = 3.20, 3.56$ and $3.93$. This enables us to probe scales from $k \geq 0.001~s/\mathrm{km}$ to $k \leq 0.02$ and $k \leq 0.07~ s/\mathrm{km}$ respectively. \\

%Modeling the flux power spectrum requires solving the non-linear regime of structure formation and the intergalactic gas in the cosmological hydrodynamics simulations that are used to that effect. I controlled for several of many systematic uncertainties related to the simulations. First, I ran simulations with different initial conditions to quantify the sampling variance. I then tested the accuracy of a splicing technique that we use to construct the flux power spectrum from lower size and lower resolution simulations. This required producing a complete run of a $(100~h^{-1}\mathrm{Mpc})^3$ comoving cube containing $2 \times 2048^3$ dark matter particles and baryons. Moreover, I quantified the impact of the neutrino mass ordering on the flux power spectrum. This enabled our working group to enhance the previously established constraints on the sum of neutrino masses from $\sum m_\nu < 0.15~\mathrm{eV}$ to the most stringent constraint to date $\sum m_\nu < 0.12~\mathrm{eV}$ with $95\%$ confidence. I then worked on implementing right-handed neutrinos in non-cold dark matter cosmological frameworks. A substancial amount of work has gone into applying plausible initial conditions that would accurately model the free-streaming effect of these types of particles. I put the most stringest constraints (at the time of publication) on the mass of non-resonantly produced sterile neutrinos as pure warm dark matter candidates, $m_\nu \geq 25~\mathrm{keV}$ at $95\%$ confidence. I extended this investigation into a mixed warm plus cold dark matter cosmology. Finally, I implement right-handed neutrinos produced in presence of a lepton asymmetry which boosts their production and lowers their free-streaming scale. I started a collaboration with a team of theoretical physicists involved in searching for astrophysical evidence for the existance of such resonantly-produced right-handed neutrinos in dark matter rich systems. Our new-born collaboration has enabled the first ever constraints on their mass using the  Ly-$\alpha$ forest power spectrum. 
%}

\begin{textblock}{10}(2,.8)
\logoED
\end{textblock}
%\vfill


\begin{textblock}{13}(2,3)
\paragraph{Title : \PhDTitleEN} 
\paragraph{Keywords : }\PhDkeywordsEN  \bigskip

\textbf{Abstract :} \PhDsumEN
\end{textblock}


\begin{textblock}{13}(13,14.5)
\includegraphics[height=2cm]{Logo/logoEgrey.png}
\end{textblock}
\parindent=0pt 

\begin{textblock}{10}(2.2,14.6)\color{blue!20!red!45!black}{\footnotesize{\textbf{Universit\'e Paris-Saclay}\\Espace Technologique / Immeuble Discovery \\
Route de l'Orme aux Merisiers RD 128 / 91190 Saint-Aubin, France}}

\end{textblock}
\cleardoublepage