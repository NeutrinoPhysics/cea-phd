\chapter*{General Conclusion}
{\color{purple}\titlerule[2.5pt]}
\vspace{4pc}%

\begin{intro}
{\color{purple}I}nvestigating my research interest at the CEA Saclay Cosmology/BAO group has been a fulfilling and fruitfull experiece. Though the original line-of-work was to determine the mass of cosmological neutrinos, my team members and advisor were open, supportive and helpful in expanding additional and new routes. \\

In the work presented in this thesis, I used the power spectrum of the transmitted flux in the Lyman-alpha (Ly-$\alpha$) forest of distant quasars to constrain the mass of cosmological neutrinos in the context of four seperate projects. Neutrinos leave a signature imprint on large scale structures in the Universe through their free-streaming, which manifests as a deficit of matter density fluctuations on typical length scales that are inversely proportional to their rest mass. This typical free-streeming scale, of order a few Mpc, can be probed by Ly-$\alpha$ forests which are imprints of the neutral atomic Hydrogen density along the background quasar's line-of-sight. I use the Ly-$\alpha$ flux power spectrum from mainly two large scale structure surveys: the $13,821$ low-resolution quasar spectra from the ninth data release of SDSS (BOSS) in 12 redshift bins from $\langle z \rangle = 2.2$ to $4.4$; and the $100$ high-resolution quasar spectra from the XQ-100 survey (of the VLT's XShooter spectrograph) in 3 redshift bins, $\langle z \rangle = 3.20, 3.56$ and $3.93$. This enables us to probe scales from $k \geq 0.001~s/\mathrm{km}$ to $k \leq 0.02$ and $k \leq 0.07~ s/\mathrm{km}$ respectively. \\

Modeling the flux power spectrum requires solving the non-linear regime of structure formation and the intergalactic gas in the cosmological hydrodynamics simulations that are used to that effect. I controlled for several of many systematic uncertainties related to the simulations. First, I ran simulations with different initial conditions to quantify the sampling variance. I then tested the accuracy of a splicing technique that we use to construct the flux power spectrum from lower size and lower resolution simulations. This required producing a complete run of a $(100~h^{-1}\mathrm{Mpc})^3$ comoving cube containing $2 \times 2048^3$ dark matter particles and baryons. Moreover, I quantified the impact of the neutrino mass ordering on the flux power spectrum. This enabled our working group to enhance the previously established constraints on the sum of neutrino masses from $\sum m_\nu < 0.15~\mathrm{eV}$ to the most stringent constraint to date $\sum m_\nu < 0.12~\mathrm{eV}$ with $95\%$ confidence. \\

I then worked on implementing right-handed neutrinos in non-cold dark matter cosmological frameworks. The $\nu$MSM is an extension of the standard model of particle physics that adds three right-handed neutrinos, insensitive to the weak interaction, to the three lepton-flavoured left-handed ones of the SM. Together, they provide a mechanism for non-zero neutrino masses, lepton and baryon asymmetries and a dark matter candidate particle. The latter one was the subject of my interest.  A substancial amount of work has gone into applying plausible initial conditions that would accurately model the free-streaming effect of these types of particles. I put the most stringest constraints (at the time of publication) on the mass of non-resonantly produced sterile neutrinos as pure warm dark matter candidates, $m_\nu \geq 25~\mathrm{keV}$ at $95\%$ confidence. I extended this investigation into a mixed warm plus cold dark matter cosmology. Finally, I implement right-handed neutrinos produced in presence of a lepton asymmetry which boosts their production and lowers their free-streaming scale. I started a collaboration with a team of theoretical physicists involved in searching for astrophysical evidence for the existance of such resonantly-produced right-handed neutrinos in dark matter rich systems. Our new-born collaboration has enabled the first ever constraints on their mass using the  Ly-$\alpha$ forest power spectrum. \\

As for the scientific aspect of this general conclusion, I am confident to affirm that if one assumes that Dark Matter is made of elementary particles, then neutrinos cannot be the sole species making up the $\Omega_\mathrm{dm}=0.26$ of the Universe's energy critical density, unless they are several tens of keV heavy, or are produced in presence of a consequential lepton asymmetry. X-ray bounds from DM-rich galaxy clusters are inconsistent with such heavy sterile neutrinos. The more plausible case is that dark matter is an admixture of neutrinos (either lepton-charged only or with sterile neutrinos) with another type of particles. Axions are an increasingly plausible dar matter candidate, and the suite of hydrodynamics simulations I developped can be further altered to correctly implement these particles. As for the lepton-flavoured neutrinos, making up about $1\%$ of the total dark matter population, our upper bounds on $\sum m_\nu$ is just $30~\mathrm{meV}$ shy of excluding \textit{de facto} the inverted mass ordering, which has to verify $\sum m_\nu \geqslant 0.11~\mathrm{eV}$, thereby determining the absolute scale of the mass eigenstates. If Ly-$\alpha$ forests are to provide stringer constraints, it is again the hydrodynamical simulations that constitute the main bottleneck of our study through the modelling of the IGM thermal state. Numerical simulations have been a staple tool for precision cosmology in the last two decades. In the coming years, they may provide key insights into the warmth of dark matter and the mass of cosmological neutrinos.
\end{intro}