\begin{textblock}{10}(2,.8)
\logoED
\end{textblock}
%\vfill


\begin{textblock}{13}(2,3)
\paragraph{Titre : \PhDTitleFR} 
\paragraph{Keywords : }\PhDkeywordsFR  \bigskip

\textbf{R\'esum\'e :} \PhDsumFR 
\end{textblock}


\begin{textblock}{13}(13,14.5)
\includegraphics[height=2cm]{Logo/logoEgrey.png}
\end{textblock}
\parindent=0pt 



\begin{textblock}{10}(2.2,14.6)\color{blue!20!red!45!black}{\footnotesize{\textbf{Universit\'e Paris-Saclay}\\Espace Technologique / Immeuble Discovery \\
Route de l'Orme aux Merisiers RD 128 / 91190 Saint-Aubin, France}}

\end{textblock}







%\newcommand{\PhDTitleFR}{D{\'e}termination de la Masse des Neutrinos Cosmologiques avec les For{\^e}ts Lyman-alpha} %% Titre en Français
%\newcommand{\PhDkeywordsFR}{Neutrinos, Mati\`ere Noire, Lyman-alpha, Cosmologie, SDSS/BOSS} %% 3 à 6 mots clefs
%\newcommand{\PhDsumFR}{Les travaux pr\'esent\'es dans cette th\`ese contraignent la masse des neutrinos dans le contexte de 4 mod\`eles de mati\`ere noire en utilisant le spectre de puissance du flux transmit dans les for\^ets Lyman-alpha des quasars distants. Les neutrinos laissent une emprunte sur les grandes structures dans l'Univers \`a travers l'\'echelle \`a laquelle ils diffusent, qui se manifeste comme un d\'eficit de fluctuations de densit\'e de mati\`ere sur des distances inversement proportionnelles \`a leur masse. De l'ordre de quelques Mpc, ces \'echelles peuvent \^etre sond\'ees par les for\^ets Ly-$\alpha$ qui tracent la densit\'e d'Hydrog\`ene neutre atomique suivant la ligne de vis\'ee du quasar en arri\`ere-plan. J'utilise le spectre de puissance Ly-$\alpha$ construit gr\^ace \`a deux relev\'es de grandes structures:les $13,821$ spectres optiques de quasars basse-r\'esolution de la 9\`eme publication des donn\'ees du SDSS/BOSS \`a 12 redshifts de $\langle z \rangle = 2.2$ \`a $4.4$; ainsi que la centaine de spectres de quasar haute-r\'esolution du relev\'e XQ-100 du VLT à $\langle z \rangle = 3.20, 3.56$ et $3.93$. Ces deux relev\'es nous permettent de sonder les \'echelles de $k \geq 0.001~s/\mathrm{km}$ \`a $k \leq 0.02$ et $k \leq 0.07~ s/\mathrm{km}$ respectivement. \\

%Mod\'eliser le spectre de puissance Ly-$\alpha$ n\'ecessite r\'esoudre le r\'egime non-lin\'eaire de formation des structure et mod\'eliser le gaz inter-galactique dans les simulations cosmologiques hydrodynamiques destin\'ees \`a cet effet. Je contr\^ole pour plusieurs incertitudes syst\'ematiques li\'ees \`a ces simulations. Dans un premier temps, je quantifie la variance d'\'echantillonnage à l'aide de simulations tourn\'ees avec diff\'erentes conditions initiales. Dans un second temps, je teste la validit\'e d'une m\'ethode permettant de construire le spectre de puissance \`a partir de simulations plus petites et moins r\'esolues. Pour ce, j'ai tourn\'e une simulation \'evoluant $2 \times 2048^3$ particules de mati\`ere noire et de baryons dans un covolume de $(100~h^{-1}\mathrm{Mpc})^3$. Dans un troisi\`eme temps, j'ai quantifi\'e l'impact de la hi\'erarchie des masses sur le spectre de puissance. Ce travail a permit \`a notre groupe d'am\'eliorer les contraintes sur la masse des neutrinos de $\sum m_\nu < 0.15~\mathrm{eV}$ \'etablie pr\'ec\'edemment \`a $\sum m_\nu < 0.12~\mathrm{eV}$ \`a $95\%$ de vraisemblance. J'ai ensuite tourn\'e mes efforts vers l'impl\'ementation de neutrinos st\'eriles en tant que candidats mati\`ere noire non-froide dans les simulations. En particulier, j'ai produit les contraintes les plus fortes (au moment de la publication) sur la masse des neutrinos st\'eriles en tant que mati\`ere noire ti\`ede: $m_\nu \geq 25~\mathrm{keV}$ \`a $95\%$ de vraisemblance. J'ai \'etendu l'\'etude dans le contexte d'une matière noire mixte et contraint l'abondance relative de la composante ti\`ede par rapport \`a la froide. Enfin, j'ai compl\'et\'e ce travail en permettant une r\'esonance dans la production des neutrinos st\'eriles, r\'eduisant ainsi leur \'echelle caract\'eristique de diffusion et refroidissant la matière ti\`ede qu'ils incorporent. \`A ce but, j'ai initi\'e une collaboration avec une \'equipe de physiciens th\'eoriciens impliqu\'es dans les recherches astrophysiques de ces neutrinos st\'eriles dits produits par r\'esonance dans des objets riches en mati\`ere noire. Notre jeune collaboration a \'etabli les premi\`eres contraintes sur leur masse en utilisant le spectre de puissance Ly-$\alpha$.}

%\begin{textblock}{10}(2,.8)
%\logoED
%\end{textblock}



%\begin{textblock}{13}(2,3)
%\paragraph{Titre : \PhDTitleFR} 
%\paragraph{Keywords : }\PhDkeywordsFR  \bigskip

%\textbf{R\'esum\'e :} \PhDsumFR 
%\end{textblock}


%\begin{textblock}{13}(13,14.5)
%\includegraphics[height=2cm]{Logo/logoEgrey.png}
%\end{textblock}
%\parindent=0pt 

%\begin{textblock}{10}(2.2,14.6)\color{blue!20!red!45!black}{\footnotesize{\textbf{Universit\'e Paris-Saclay}\\Espace Technologique / Immeuble Discovery \\
%Route de l'Orme aux Merisiers RD 128 / 91190 Saint-Aubin, France}}

%\end{textblock}