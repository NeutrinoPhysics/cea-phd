\begin{intro}
{\color{purple}C}e travail de recherche fut possible gr{\^a}ce au soutien de nombreuses personnes. Tout d'abord, je remercie Nathalie Palanque-Delabrouille pour m'avoir guid\'e tout au long de cette aventure. Je tiens \`a souligner la claret\'e et la justesse de ses connaissances et sa transmission. Je lui suis \'egalement reconnaissant de m'avoir octroy\'e la libert\'e et l'autonomie de poursuivre des pistes qui n'ont pas toujours \'et\'e concluantes, qui est pour moi une composante primordiale de la recherche, et qui a fait de mon interaction avec le groupe cosmologie de l'Irfu un v\'eritable \'echange entres pairs. Je tiens à saluer toute l'\'equipe des chercheurs permanents, et plus particuli\`erement Christophe Y\`eche pour m'avoir pr\'esent\'e le sujet et le CEA. Je garderai des pens\'ees particuli\`eres pour James Rich pour les discussions sur la riche diversit\'e de prononciation anglo-saxone, Etienne Burtin pour ses fr\'equentes visite du bureau 36 qui ont approfondi ma culture cin\'ematographique, Eric Armengaud pour les \'echanges fructueux et les frustrations partag\'ees sur le cluster, Christophe Magneville pour son aide pr\'ecieuse sur les simulations num\'eriques, Jean-Marc Le Goff et Jean-Baptiste Melin pour les discussions techniques et les remarques cibl\'ees (surtout en r\'eunion de groupe), Vanina Ruhlmann-Kleider pour son impeccable p\'edagogie et implication dans l'\'ecole doctorale, et enfin Laurent Chevalier pour m'avoir parain\'e tout au long de mon s\'ejour tant \`a Saclay qu'\`a Trieste. \\

Le doctorat est \'egalement une aventure humaine, et je tiens \`a remercier les ``non-permanents'' de l'Irfu pour les trois ann\'es fantastiques avec qui j'ai eu l'honneur et la joie de partager. Les fous rires avec Pauline Zarrouk vont beaucoup me manquer, ainsi que les petites siestes et go\^ut\'es du 4h au bureau 36. J'ai \'enorm\'ement appr\'eci\'e la profondeur, culture et maturit\'e d'H\'elion du-Mas-des-Bourboux, avec qui j'ai pu d\'evelopper une amiti\'e qui m'est ch\`ere. Merci \'egalement \`a Pierre Laurent, Ana Niemiec, Pierros Ntelis, Loic Verdier, Charles-Antoine Claveau, Arnaud De-Mattia, Cl\'ement Besson et Anand Raichoor pour avoir constitu\'e la BOSS team de r\^eve, tant en conf\'erence \`a l'\'etranger que tard le soir sur les quais de la capitale ! J'en profite pour v\'ehiculer ma haine profonde pour Cl\'ement Leloup que j'esp\`ere ne plus jamais revoir. \\

Enfin, je tiens \`a remercier ma famille pour son soutien moral et sa patience pendant ces trois ann\'ees. C'est une sacr\'ee motivation d'avoir une famille admiratrice et curieuse de mon travail, surtout pendant les moments difficiles. Merci aussi \`a Roxane Barnab\'e pour les mois inoubliables que j'ai pass\'e en ta compagnie, et pour la patience dont tu as fait preuve lors de la p\'enible p\'eriode de r\'edaction. Je salue mes amis avec qui j'ai pass\'e de (trop) nombreuses soir\'ees et weekends \`a pouvoir \'echapper mon quotidien ! \\

Ce travail a b\'en\'efici\'e des allocations de ressources RA-2371, PA-2777 et RA-7706 de \texttt{PRACE} et \texttt{GENCI} , donnant acc\`es \`a un total de 11.9 millions d'heures de calcul sur les machines du \texttt{TGCC}/\texttt{CCRT}. 
\end{intro}