\vspace*{3pc}
\begin{center}
\begin{minipage}{0.7\linewidth}
\hrule
\vspace{8pt}
{\huge\guillemotleft} ~The light shines in the darkness, and the darkness has not overcome it. {\huge\guillemotright}  \\
\vspace{2pt}
\begin{flushright}
--- \textsc{John 1:5}
\end{flushright}
\vspace{8pt}
\hrule
\end{minipage}
\end{center}
\vspace{3pc}

%\section*{Introduction}
\begin{intro}
{\color{purple}I}n the previous chapter, I detailed the impact of neutrino mass on the matter power spectrum. This was under the assumption that one can linearly perturb the density fields around their background value. However, since $\delta$ increases monotonically due to the attractive nature of the gravitational interaction, the approximation of linear theory eventually breaks down below some scale $k_0$. Linear theory remains valid as a perturbative treatment for $k < k_0$ where \\
\end{intro} 
\begin{equation*}
\begin{array}{c}
k_0^3 ~\vert \delta(k_0) \vert^2 \sim 1 \\
\Leftrightarrow \\
\Delta^2 (k_0) \sim 1 \\
\Leftrightarrow \\
\sigma^2 (1/k_0) \sim 1
\end{array}
\end{equation*} \\ For higher $k$ values, the evolution of $\delta$ is said to be \emph{non-linear}. One must explicitely solve the gravitational interactions between particles with numerical simulations. This is layed out in the following chapter. In the present chapter, I lay out how the power spectrum can be measured in the non-linear regime. In Sec.~\ref{sec:p1d}, I introduce quasi-stellar objects, which are the brightest sources of light in the Universe. The spectrometric properties of their light observed with Earth telescopes give us valuable information on their environment through their atomic spectral emission lines. The absorption lines, on the other hand, yield information on the intergalactic medium in which their light travels. A particular region of interest is the region blueward of the Lyman-alpha emission line, which entails the redshift and density of Hydrogen absorbers along the quasar line-of-sight. Known as the Lyman-alpha forest, it is a very active area of research for both extragalactic astrophysicists and cosmologists. I also introduce the main observable for this work: the power spectrum of the transmitted flux fraction in the Lyman-$\alpha$ forest of quasars. In Sec.~\ref{sec:pfdata}, I describe three sets of quasar samples taken from large scale structure surveys, from which I compute the Lyman-alpha power spectrum. 
