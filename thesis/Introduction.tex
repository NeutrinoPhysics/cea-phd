\chapter*{General Introduction}
{\color{purple}\titlerule[2.5pt]}
\vspace{4pc}%

\vspace*{3pc}
\begin{center}
\begin{minipage}{0.7\linewidth}
\hrule
\vspace{8pt}
{\huge\guillemotleft} ~We live on an island of knowledge surrounded by a sea of ignorance. As our island of knowledge grows, so does the shore of our ignorance. {\huge\guillemotright} \\
\vspace{2pt}
\begin{flushright}
--- \textsc{John Archibald Wheeler}
\end{flushright}
\vspace{8pt}
\hrule
\end{minipage}
\end{center}
\vspace{3pc}


\begin{intro}
{\color{purple}T}rying to convince anyone that neutrinos are interesting can be a challenge. Disclose that they switch flavour or that they may break charge parity symmetry, excitement isn't the standard look you'll get. Whenever I put forth that they may serve as the gravitational scaffolding that bounds the largest objests in the Universe together ( galaxies and clusters of galaxies ), that boredom, indifference or confusion usually turns into bewilderment, skepticism and awe. How can the tiniest of particles --- so tiny that you'll need a lead wall a lightyear thick to have a reasonable chance to stop a single one in its tracks --- be so influencial on the largest, literally astronomical scales ? Disclosing that this is precisely what I am investigating as part of my PhD research triggers a salve of thoughtful and keen questions, to which I am rarely in the position to assert that we know anything for sure. \\

The answer to whether or not these \textit{ghost} particles could be the long-sought dark matter really comes down to how massive they are. So why not just weigh them in the laboratory? The catch is, neutrinos are ridiculously light. If you were to weigh the combined mass of every neutrino that has traversed every human being from birth to death ever since the dawn of mankind, you would not even get $150$ milligrams. How could we measure a mass so small? Well, the silver lining is in that very thought experiment. Even though an individual neutrino is light and evasive, they are so numerous in the Universe that they are second only to photons, number-wise. On Earthly distances, the effect of their mass is insignificant. On galactic and extra-galactic distances however, you may have just about enough to influence the gravitational dynamics of these systems. The key in figuring out their mass is to find a galactic-sized bathroom scales. \\

As with everything else in astronomy, we can only study the nature and properties of objects from the light they shine on us. But the Universe is essentially dark, especially in the mostly empty space in between galaxies, where neutrino mass is expected to matter. The light from the most distant and most luminous objects in the Universe, produced from accreting supermassive black holes at the center of galaxies, shines through the tenous gas in the intergalactic medium, revealing the distribution of this otherwise obscured matter in between galaxies akin to a streetlight shining through a thick fog. Directing this light onto a prism, unraveling the different colours that compose it, astronomers learn properties of the foggy foreground from the frequencies of light it occults from the background light. The absorption features imprinted on the spectrograph of these background lights are impacted by the mass of neutrinos. \\

This manuscipt is arranged into 6 chapters. The \ref{chap:neutrinos}st chapter specifies some general knowledge about neutrinos and their properties. To study them at the scales of interest, I place them in the context of cosmology in the \ref{chap:cosmology}nd chapter and describe how they impact the formation of large scale structures in the \ref{chap:structure}rd chapter. In the \ref{chap:LyaForest}th chapter, I describe how the intergalactic ``fog'' serves as our main observational tool to measure the mass of neutrinos, and how our data set is constructed. In the \ref{chap:Simulations}th chapter, I detail how we compute the expected distribution of intergalactic matter based on selected cosmological and astrophysical ingredients using numerical simulations. Chapter \ref{chap:results} compiles the results of the analysis that compares our predictive models from observations on the mass of neutrinos and whether or not they make up dark matter.

\end{intro}

