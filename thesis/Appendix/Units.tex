\chapter{System of Units}
\label{apx:units}
{\color{purple}\titlerule[2.5pt]}
\vspace{4pc}%

A unit is dividing a physical measurable quantity by a standard. When one says an object is $3~\mathrm{m}$ long, what it means is that dividing the length of the object by one meter yields the value $3$. A good system of unit is one in which the absolute value of the physical quantities usually measures a number between $0.01$ and $100$. The \textit{syst\`eme international}, or SI, is a system of units taylored to practical, everyday human life. It is no coincidence that the average height of a person is of order unity in meters, or that their mass is several tens of $\mathrm{kg}$. In astrophysics and cosmology, the SI is poorly suited system of units, as expressing luminosities, masses, time intervals, speeds, \textit{etc}, in their SI units would yield large powers of ten. A more suited unit of distance in extragalactic astrophysics is the megaparsec ($\mathrm{Mpc}$), \textit{i.e.} the typical distance between two galaxies today
\begin{equation*}
\mathrm{Mpc}~=~ 10^6~ \mathrm{pc} ~=~ 3.086 \times 10^{19}~km
\end{equation*} A typical unit of frequency (inverse of time) for cosmological systems is the Hubble parameter
\begin{equation*}
H~=~100~h~\mathrm{km}~s^{-1}\mathrm{Mpc}^{-1} ~= ~3.24 \times 10^{-18}~h ~s^{-1}
\end{equation*} where $h$ is the expansion rate of the Universe in units of $100~\mathrm{km}~s^{-1}\mathrm{Mpc}^{-1}$, typically $h \simeq 0.7 \pm 0.1$. Another suited system of units is expressing velocities in units of $c$, the speed of light in vaccum; actions in units of $h_p$, the Planck constant; energies per temperature in units of $k_b$, the Boltzmann constant, and accelerations in units of $G$, the Newton gravitational constant. \\

To ``convert'' the measures from the SI to this $c=h_p=k_b=G=1$ system of units, first express the fundamental constants in their SI units:

\begin{equation*}
\begin{array}{ccll}
G & = & 6.674 \times 10^{-11} & \mathrm{m}^3 \mathrm{kg}^{-1} s^{-2}\\
c & = & 2.998 \times 10^8 & \mathrm{m}~s^{-1}\\
k_b & = & 1.381 \times 10^{-23} & \mathrm{m}^2 \mathrm{kg}^{-1} s^{-2} \mathrm{K}^{-1}\\
h_p & = & 6.626 \times 10^{-34} & \mathrm{m}^2 \mathrm{kg}~s^{-1}
\end{array}
\end{equation*} \\ Expressing the Joule, the SI unit of energy, in units of electronvolts ($\mathrm{eV}$)
\begin{equation*}
J~=~\mathrm{m}^2\mathrm{kg}~s^{-2}~=~6.242 \times 10^{18}~\mathrm{eV}
\end{equation*} enables one to express distances, masses, temperatures and mass densities in the new system of units:

\begin{equation*}
\begin{array}{ccll}
\mathrm{m} & = & 2.418 \times 10^6 & h_p c ~\mathrm{eV}^{-1} \\
\mathrm{kg} & = & 6.242 \times 10^{34} &c^{-2} ~\mathrm{eV} \\
\mathrm{K} & = & 8.620 \times 10^{-5} & k_b^{-1} ~\mathrm{eV} \\
\mathrm{kg}~\mathrm{m}^{-3} & = & 4.416 \times 10^{15} & h_p^{-3}c^{-5}\mathrm{eV}^4
\end{array}
\end{equation*}