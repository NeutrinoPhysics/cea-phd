\vspace*{3pc}
\begin{center}
\begin{minipage}{0.7\linewidth}
\hrule
\vspace{8pt}
{\huge\guillemotleft} ~To ask the right question is harder than to answer it. {\huge\guillemotright} \\
\vspace{2pt}
\begin{flushright}
--- \textsc{Georg Cantor}
\end{flushright}
\vspace{8pt}
\hrule
\end{minipage}
\end{center}
\vspace{3pc}


\begin{intro}
{\color{purple}I}n the previous chapter, I've introduced the basic framework in which cosmological neutrinos are treated. Setting aside dark energy, the takeaway message is that the Universe can be thought of as a bath of interacting particles that expands adiabatically. The challenge that sets the cosmological fluid apart from a regular self-gravitating fluid is the fact that there is no inertial reference frame in which one can express the conservation of mass or energy and momentum due to a homogenous and isotropic expansion of all points of space. The trick used to curtail this oddity is to establish a \emph{comoving} coordinate system, that is not anchored to any reference frame at all. This enabled us to relate comoving quantities of the background. This artificial framework left a rather singular signature: immobile observers in this comoving coordinate system disagree on the energy of photons, so far the only\footnote{that is besides gravitational waves and neutrinos} messengers of objects and processes in the cosmos, and the amount by which it is redshifted is intrinsically linked to the expanding geometry of spacetime. Nevertheless, the fluid approximation at rest in this coordinate system enabled us to greatly simplify the Einstein field equations which link the local curvature of spacetime with its energy density. We abused these equations to get a general idea of how the entire cosmic background contents behave globally, which gave us the Friedmann equations, attesting that the expansion rate and its rate of change are driven by the total energy density content. \\

The anthropic principle, \textit{i.e.} the fact that you are presently reading this PhD thesis, points to the obvious fact that this simplification is incomplete in describing the rich and diverse array of acoustic, chemical, electromagnetic, and nuclear processes happening around us. The Universe cannot be uniformely homogenous. To account for structures, large and small, we must look at what happens to \emph{inhomogeneities} when they arise. In Sec.~\ref{sec:PS}, I lay out the formal statistical definition of the tool used to probe such inhomogeneities, the \textbf{power spectrum}, and describe how it can be used to quantify the clustering of large scale structures. I then guide the reader on how one can deduce its shape, at least on large scales, from linear perturbations theory in the context of the expanding Universe in Sec.~\ref{sec:LP}. This section can be skipped on first reading, as it is technical and I haven't made any worthwhile contributions in the theory, besides grouping together different ideas and condensing books-worth of information into a single dense section. It is written with the intention of providing a reader longing to grasp the fundamental underpinnings of linear perturbation theory and the basics of how a Boltzmann solver code works. The conventions adopted and argumentation are adapted from and heavily influenced by the NPAC\footnote{\url{http://npac.lal.in2p3.fr/home/}} Master's advanced lectures on Cosmology by \textit{Michael Joyce} (LPNHE -- Paris VII U.), \textit{Mathieu Langer} (IAS -- Paris XI U.) and \textit{Herv\'e Dole} (IAS -- Paris XI U.), \textsc{Modern Cosmology}~\citep{DodelsonBook} and References \cite{MaBe94, LESGOURGUES2006, BLR09, Wong2011, Pastor2011, LP2012, Abaz_review_2017}. \\


Neutrinos are the lightest\footnote{of all particles having a rest mass} known particle having been discovered. Their elusiveness makes it challenging to weight them in laboratory, accelerator or reactor experiments. The background, however, has expanded to sufficiently low temperatures that all neutrinos produced in nuclear reactions in the early Universe have become non-relativistic. Neutrinos are therefore a rather singular component as they've contributed as additional radiation for a substantial portion of the age of the Universe, contributing to the expansion rate, before their small masses became relevant enough for them to cluster and behave like a self-gravitating matter component. Because their non-relativistic transition likely happened some time during the matter-dominated era, they left a characteristic imprint on the distribution of matter at intergalactic scales that is dependant on their mass. This small scale signature on the matter power spectrum is therefore the main observable for constraining neutrino masses in cosmology studied throughout this work. I characterize the free-streaming length scale of neutrinos and other non-cold dark matter particles and illustrate how their mass influences the power spectrum at small scales in Sec.~\ref{sec:CAMB}. I also describe how linear perturbations are solved numerically with some widely-used open-access Boltzmann codes, which have been used to implement neutrinos and non-cold dark matter components in the simulations I describe in chapter~\ref{chap:Simulations}. Indeed, the scales of concern are expected to be dominated by non-linear effects such as gravitational back-reactions, coupling to other species, baryon effects, thermal broadening and feedback mechanisms, which limit the predictions of linear theory. This compels one to consider a more adequate observable in the non-linear regime: the one-dimensional power spectrum of the Lyman-alpha forest, which I defer to the next chapter.
\end{intro}
