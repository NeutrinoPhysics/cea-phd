\vspace*{3pc}
\begin{center}
\begin{minipage}{0.7\linewidth}
\hrule
\vspace{8pt}
{\huge\guillemotleft} ~Neutrino physics is largely an art of learning a great deal by observing nothing. {\huge\guillemotright} \\
\vspace{2pt}
\begin{flushright}
--- \textsc{Haim Harari}
\end{flushright}
\vspace{8pt}
\hrule
\end{minipage}
\end{center}
\vspace{3pc}


%\section*{Introduction}

\begin{intro}
{\color{purple}E}very second you spend reading this thesis, keep in mind there are approximately 65 billion neutrinos passing through each square centimeter of your body facing the Sun. Almost all of them are produced in the nuclear reactions occuring in the core of our star, some $150$ million kilometers away. In fact, the speculation and subsequent discovery of neutrinos enhanced our understanding of the way the Sun burns its Hydrogen fuel: through nuclear reactions. In the context of physics, nuclear reactions involve two main fundamental interactions: (1) the strong interaction, which confines quarks, the fundamental particles making up all of matter\footnote{hadrons}; and (2) the weak interaction, which describes how quarks and leptons interact non-elastically. Without the latter, protons ($^{1}_{1}\mathrm{H}^{+}$) would not be able to fuse into Deuterium ($^{2}_{1}\mathrm{H}^{+}$) and Tritium ($^{3}_{1}\mathrm{H}^{+}$) and we most certainly would not be here to appreciate the non-trivial role of neutrinos in the Universe, despite their seemingly inconsequential significance the first sentence of this paragraph may suggest. \\

But stellar cores aren't the only neutrino factories in the Universe. Weak interactions are also involved in radioactive decay of certain atomic nuclei, which happen virtually anywhere there is baryonic matter, but more preponderantly: \\

\begin{itemize}
\item[$\bullet$] in the Earth's interior, producing so-called \textit{geological} neutrinos, \\
\item[$\bullet$] in upper atmosphere particle's interactions with cosmic rays, producing so-called \textit{atmospheric} neutrinos, \\
\item[$\bullet$] in fission and fusion bombs and nuclear reactors, producing so-called \textit{reactor} neutrinos, \\
\item[$\bullet$] in man-made particle accelerators, producing so-called \textit{accelerator} neutrinos, \\
\item[$\bullet$] in supernov{\ae} explosions, producing so-called \textit{astrophysical} neutrinos, \\
\item[$\bullet$] and finally, in the hot plasma in the early Universe, producing so-called \textit{cosmological} neutrinos. \\
\end{itemize} 

All but the last have been confirmed sources of neutrinos. Cosmological neutrinos produced after the Big Bang are expected to exist based on our current understanding of particle physics and our cosmological model, but in far less preponderant quantities. The cosmic neutrino background hasn't yet been detected, but it is estimated that at any given time, there are approximately $112$ neutrinos and anti-neutrinos of each lepton charge (electronic, muonic and tauic) depending on which charged lepton they interact with via the weak force. Nevertheless, cosmological neutrinos may be key to solve one of physic's long-lasting mystery: dark matter. The role of neutrinos in the context of our cosmological framework is explicited in Chap.~\ref{chap:cosmology} and Chap.~\ref{chap:structure}. In this current chapter, I recap the main properties of neutrinos. In Sec.~\ref{sec:sm}, I describe them in the context of weak interactions, and how they can be distinguished into three different lepton flavours. In Sec.~\ref{sec:bsm}, I describe the mechanism of neutrino flavour oscillations with an analogy, and argue why the observation of this phenomenon proves at least one of the three lepton-charged neutrinos has non-zero mass. In Sec.~\ref{sec:numsms}, I introduce hypothetical \textit{sterile} neutrinos which carry no lepton charge, and may be a more viable candidate particle for the dark matter in the Universe than their weak-sensitive counterparts. 

\end{intro}

